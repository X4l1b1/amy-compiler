The following example describe two implementations of the same \texttt{Factorial} function in \textsl{Amy}. Although both examples require similar amount of code, it will be explained how both lead to different results and how their respective process flow is impacted.

\begin{lstlisting}
/* Implementation using previously defined features */
object Factorial {
  def fact(i: Int) : Int = { 
    if(i < 2) { 1 }
    else {
        i * fact(i - 1)
    }
  }
}
\end{lstlisting}
This rather simple function compute the factorial value of a given \texttt{Integer}. This function presents a recursive structure and will call itself as many times as required before the last call returns the value \texttt{1} and make all the other calls return like a waterfall. Each iteration, except the last one, is composed of 1 comparison, 1 subtraction, 1 multiplication and 1 function call.
\newpage
\begin{lstlisting}
/* Implementation using imperative features */
object Factorial {
  def fact(i: Int) : Int = { 
    var fact: Int = 1;  // Declaration
    var index: Int = 2;
    while(index <= i){
        fact = fact*index; // Reassignment
        index = index + 1
    }
    fact    
  } 
}
\end{lstlisting}
In this implementation, the \texttt{var} keyword is used to define two mutable \texttt{Integer} variables with initial values \texttt{1} and \texttt{2} respectively. Then the variable \textsl{index} is compared to the function's argument to express the current state of the function's processing. If \texttt{true}, the processing requires at least another step : the \texttt{fact} variable is updated and the \texttt{index} is incremented, the condition (\texttt{index <= i}) is then evaluated again. When \texttt{false}, the function goes to the next line after the \texttt{while} instruction's body and return the current value of \texttt{fact}.
This implementation, as opposed to the first one, requires a single call to the function \textsl{fact} but requires the same number of steps, each including 1 comparison, 1 addition and 1 multiplication. \\

The mutable variables are declared in a similar way the immutable ones are, using only the keyword \texttt{var} instead. On the other hand, in the reassignment of those variables, a whole new usage of the "=" (\texttt{EQUALS} token) delimiter is made, it is used as an operator.

The condition provided after the \texttt{while} instruction is evaluated first, if it is \textsl{true}, then the associated block is executed and an unconditional branch leads back to the condition for another evaluation, otherwise, the program jumps directly to the line following the aforementioned block.\\


From those examples, both aspects of those implementations can be directly compared : in terms of algorithmic complexity, both require more or less the same amount of similar operations but in a more concrete perspective on the compiled result, we see that the second implementation, using imperative features, requires only 1 call to \textsl{fact} and only 2 variable assignments regardless of the argument's value. However, the first example, purely functional, requires a linear amount of call to \textsl{fact} with respect to the value of \texttt{i}.
